\documentclass[titlepage,a4paper]{article}

\usepackage{amsmath}
\usepackage{float}
\usepackage{caption}
\usepackage{booktabs}



\begin{document}

    \setlength{\parindent}{0pt}
    \setlength{\parskip}{0.5em}

    \subsection{Question 1(c) B}
    

        \subsubsection{Assumption}

        For Question 1(c), the following assumptions are made:

            \begin{enumerate}
                \item Only the cost of maintenance and the cost of installation of air conditioners are considered.
                \item \label{Q1(c) B Assumptions 2} All the budget are not separated, and can be used flexibly after the durable years is determined. 
                \item 
            \end{enumerate}

        \subsubsection{Analysis} \label{Q1(c) B Analysis}
        
        According to assumption \ref{Q1(c) B Assumptions 2}, part of the maintenance budget can also be used to install air conditioners after the durable years has been determined. 

        Comparing with Question 1(c) A, the target keeps the same, but the constraints is changed. For $N$ years, the cost for installation and maintenance during $N$ years should be considered together as shown in the following equation.

            \begin{equation}
                \sum_{n=1}^N \quad (C_{m,x}(n)+3000)x+(C_{m,y}(n)+1500)y \leq 24000+300E_1+(4000+100E_1)N 
            \end{equation}

        \subsubsection{Model}

        According to \ref{Q1(c) B Analysis}, the model of Question 1(c) can be made as following:

        For N durable years:

        \begin{equation}
            \begin{aligned}
             &\min_{x,y}  -(4x +2.5y) \\                 
             & s.t.
            \begin{cases}
                \begin{aligned}
                    x+y & \leq & 12 \\
                    C_{inst}+\sum_{n=1}^N C_{m,x}(n)x+C_{m,y}(n)y & \leq & 24000+300E_1+(4000+100E_1)N \\
                    x,y  & \geq & 0 \\
                \end{aligned}
            \end{cases}                
            \end{aligned}
        \end{equation}

        

        \subsubsection{Solution}

        Different durations lead to different amount of maintenance budget can be used as installation budget, so the analysis to simplify the calculation in Question 1(c) A cannot be done anymore. We just calculate the maximal value for different durable years separately and find the optimal one from them. 

        
        Based on equation in, the corresponding parameters in MATLAB function \verb quadprog are shown as follows:

        % $$

        % $$

        For years ranging from 1 to 10, the installation plan and the maximum power are found as shown in Table \par

        \begin{table}

            \centering

            \caption{Results for different duration time (non-integrization)}
            \begin{tabular}{*{11}{c}}

                \toprule

                duration & 1 & 2 & 3 & 4 & 5  \\

                \midrule

                max power & 40.6212 & 42.9515  & 44.8285 & 45.6772 & 46.4063   \\

                X amount &  7.0808  & 8.6344  & 9.8857 & 10.4515 & 10.9375  \\

                y amount &   4.9192  &  3.3656  &  2.1143  &  1.5485  & 1.0625   \\

                \bottomrule

                \toprule

                duration  & 6 & 7 & 8 & 9 & 10 \\

                \midrule

                max power  & 46.3235 & 45.6208 & 44.4924 & 43.1050 & 41.5873 \\

                X amount  & 10.8824 & 10.4139 & 9.6616 & 8.7367 & 7.7249 \\

                y amount  & 1.1176 &  1.5861 & 2.3384 & 3.2633 & 4.2751 \\

                \bottomrule


            \end{tabular}
        \end{table}

        Without regard to integration, The optimal choice for duration time is 5 year, with 10.9376 X type, 1.0625 Y type and maximum power 46.4064 (kW). The next step is adjusting this result to integer and checking whether it is still the optimal answer.

        
        


        \subsubsection{Answer}



    
\end{document}
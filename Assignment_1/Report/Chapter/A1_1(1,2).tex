\documentclass{article}
\usepackage[utf8]{inputenc}
\usepackage{amsmath}

\title{OSC_A1}
\author{liyiting.kemos }
\date{October 2020}

\begin{document}



\section{Task1}
    \subsection{Task1.(a)}
        \subsubsection{Assumption}
            A maximum linear programming problem was described in this question. We assume that:\\
            \begin{enumerate}
                \item We will install central air conditioner K, the number of which is x, and split-type air conditioner Y, the number of which is y. 
                \item We only install air conditioners once.
                \item Although x and y is integer, we consider them as non-negative real number first.
            \end{enumerate}
    
        \subsubsection{Analysis}
            We can easily find that there are two constrains, namely, budget constrains and amount constrains. \\
            \begin{enumerate}
                \item We can not install more than 12 air conditioners or install a negative number of air conditioners. 
                \item Total budget for all installation is $24000 + 300E1$ Euro, where $E1 = 9$.
            \end{enumerate}
            
        \subsubsection{Model}
            According to analysis above, we can formalize  this optimization problem, we have:
            \begin{align}\label{1.1}
            \begin{split}  
                max_{x,y} &\quad 4*x + 2.5*y \\
                s.t.\qquad x + y &\leq 12 \\
                3000*x + 1500*y &\leq 24000 + 300*E1 \\
                x,y &\geq 0 \\
            \end{split}    
            \end{align}      
            Obviously, model above isn't a standard form of linear programming problem, we can transform it into standard form, and we have:
            \begin{align}\label{1.2}
            \begin{split}  
                -min_{x,y,s_1,s_2} &\quad -4*x - 2.5*y \\
                s.t.\qquad x + y + s_1 &= 12 \\
                3000*x + 1500*y +s_2 &= 24000 + 300*E1 \\
                x,\,y,\:s_1,\,s_2 &= 0 \\
            \end{split}    
            \end{align}      
            Formula \eqref{1.2} is a standard form of LP problem described in Task1.(a).

    \subsection{Task1.(b)}
        When solving problem in Task1.(a) by using MATLAB, we will use following function:
        \begin{align}\label{1.3}
        \begin{split}  
            [x,val,flag] = linprog(c,A,b,&Aeq,beq,lb,ub,options)
        \end{split}    
        \end{align}    
        Formula \eqref{1.3} represent solution of following questions:
        \begin{align}\label{1.4}
        \begin{split}  
            min_x &\quad f*X\\
            A*X &= b\\
            Aeq*X &= beq\\
            lb\leq X &\leq ub
        \end{split}    
        \end{align} 
        In formula \eqref{1.3} $x$ represent the optimization of independent variable, $val$ represent the optimized outcome, $flag$ is a sign of whether the problem has a solution. When the $flag$ is 1, it means that the problem has an optimal solution. If the $flag$ is 0, the problem has no optimal solution.\\
        We can easily tell that, in Task1.(b),:
        \[ f = \begin{bmatrix}\label{1.5}
        -4&-2.5
        \end{bmatrix}\]
        \[ A = \begin{bmatrix}\label{1.6}
        1&1\\
        3000&1500
        \end{bmatrix}\]
        \[ b = \begin{bmatrix}\label{1.7}
        12\\
        24000+300*E1
        \end{bmatrix}\]
        \[ lb = \begin{bmatrix}\label{1.8}
        0&0
        \end{bmatrix}\]
        \[ ub = \begin{bmatrix}\label{1.9}
        inf&inf
        \end{bmatrix}\]
        \begin{align}\label{1.10}
            Aeq = beq = [\,]
        \end{align} 
        \subsubsection{Solution}
            MATLAB code, which is not shown here, will be uploaded as an attachment in the form of .m file. The final result is we install 6 $X$ air conditioners, and 6 $Y$ air conditioner, which leads to a maximum power, which is 39$kW$
            
    
    
    
\section{Task3}
    \subsection{Analysis}
        In order to have a more accurate indoor temperature discrete model, we want the finest coefficient of $a_1,a_2,a_3$. We can achieve that by minimizing the square of the error between real $T_{b,k+1}$ and estimation.\\
        From 2.1.2, we can rewrite the formula:
        \begin{align}\label{1.10}
            \sum_{i=1}^N \:\{[(T_{b,k+1} - T_{b,K}) - \Delta t * (a_1*\dot{q_{solar,k}} + a_2*(\dot{q_{occ,k}}+\dot{q_{ac,k}}-\dot{q_{vent,k}}) + a_3*(T_{amb,k} - T_{b,k}) ]  \}
        \end{align}
        We set:
        \begin{align}
        \begin{split}
            T &= (T_{b,k+1} - T_{b,K})\\
            %
            (\dot{q_{occ,k}}+\dot{q_{ac,k}}-\dot{q_{vent,k}}) &= (\dot{q_{sigma,k}})\\
            %
            T_{amb,k} - T_{b,k} &= T_{sigma,k}\\
            %
            (a_1*\dot{q_{solar,k}} + a_2*(\dot{q_{sigma,k}}) + a_3*(T_{sigma,k}) &= 
            \begin{bmatrix} \dot{q_{solar,k}}&\dot{q_{sigma,k}}&\dot{T_{sigma,k}} \end{bmatrix} * \begin{bmatrix} a_1\\a_2\\a_3 \end{bmatrix}\\
            %
            \begin{bmatrix} \dot{q_{solar,k}}&\dot{q_{sigma,k}}&\dot{T_{sigma,k}} \end{bmatrix} &= 
            \begin{bmatrix} \dot{q_k} \end{bmatrix}\\
            %
            \begin{bmatrix} a_1\\a_2\\a_3 \end{bmatrix} &= \begin{bmatrix} a \end{bmatrix}
        \end{split}
        \end{align}
        We can rewrite \eqref{1.10} to:
        \begin{align}
        \begin{split}\label{1.11}
            &\begin{bmatrix}T - \begin{bmatrix} \dot{q} \end{bmatrix} * \begin{bmatrix} a \end{bmatrix} \end{bmatrix}^T * 
            \begin{bmatrix}T - \begin{bmatrix} \dot{q} \end{bmatrix} * \begin{bmatrix} a \end{bmatrix} \end{bmatrix}\\
            &= a^T \dot{q}^T \dot{q} a - 2T^T\dot{q}a + T^TT
        \end{split}
        \end{align}
        
    \subsection{Model}
        This is a quadratic programming problem, standard form of a quadratic programming is:
        \begin{align}
        \begin{split}
            minimize_x &\quad \frac 12x^THx + c^Tx\\
            Ax &= b\\
            x &\geq 0
        \end{split}
        \end{align} 
        For formula \eqref{1.11}, it is a standard form of quadratic programming problem, where:
        \begin{align}
        \begin{split}\label{1.11}
            H &= 2 \dot{q}^T \dot{q}\\
            c^T &= 2T^T\dot{q}
        \end{split}
        \end{align} 
        
    \subsection{solution}    
        We can use function in MATLAB to solve it. Details of each matrix are store in MATLAB, as a result, we only show final result here:
        \begin{align}
        \begin{split}
           a_1 &= 2.662\times10^{-6}\\
           a_2 &= 1.056\times10^{-6}\\
           a_3 &= 2.135\times10^{-6}      
        \end{split}
        \end{align}     
        
\end{document}

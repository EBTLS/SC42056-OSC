\ifx \allfiles \undefined

\documentclass[titlepage,a4paper]{article}

\usepackage{amsmath}
\usepackage{float}
\usepackage{caption}
\usepackage{booktabs}
\usepackage{eurosym}
\usepackage[a4paper]{geometry}



\begin{document}

\fi

    \setlength{\parindent}{0pt}
    \setlength{\parskip}{0.5em}

    \subsection{Question 2}

        \subsubsection{Analysis}

        In order to transform model to a discrete-time model, we need to use the following approximation:

        \begin{equation}
            \dot{T_{b,k}} = \frac{dT_{b,k}}{dt} \approx \frac{T_{b,k+1}-T_{b,k}}{\Delta t}
        \end{equation}

        where $T_{b,k}$ represents the indoor temperature in the building at time step k.

        The target equation has derivative on both sides, but according to Question 3, the derivate of $q_{solar,k}, q_{occ,k}, q_{ac,k}, q_{vent,k}$ are able to observed directly. So only the left side of the equation needs to be discretized.

        \subsubsection{Solution} \label{solution.Q2}

        The process to obtain discrete-time model is shown as follows.

        \begin{equation}
            \begin{aligned}
                \frac{T_{b,k+1}-T_{b,k}}{\Delta t} &=a_1 \dot{q_{solar,k}}+a_2[\dot{q_{occ,k}}+\dot{q_{ac,k}}-\dot{q_{vent,k}}]+a_3[T_{amb,k}-T_{b,k}] \\
                \Rightarrow T_{b,k+1} &= \{a_1 \dot{q_{solar,k}}+a_2[\dot{q_{occ,k}}+\dot{q_{ac,k}}-\dot{q_{vent,k}}]+a_3[T_{amb,k}-T_{b,k}] \} \Delta t+T_{b,k} \\
                \Rightarrow T_{b,k+1} &= \{a_1 \dot{q_{solar,k}}+a_2[\dot{q_{occ,k}}+\dot{q_{ac,k}}-\dot{q_{vent,k}}]+a_3T_{amb,k}\} \Delta t +(1-a_3 \Delta t)T_{b,k} \\              
            \end{aligned}
        \end{equation}

        so,

        \begin{equation}
            \begin{aligned}
                A &=1-a_3 \Delta t \\
                B &=
                \begin{bmatrix}
                    a_1 \Delta t & a_2 \Delta t & a_2 \Delta t & -a_2 \Delta t & a_3 \Delta t \\
                \end{bmatrix}
            \end{aligned}
        \end{equation}
        

        \subsubsection{Answer}

        From \ref{solution.Q2}, the answer is:
        
        \begin{equation}
            \begin{aligned}
                A &=1-a_3 \Delta t \\
                B &=
                \begin{bmatrix}
                    a_1 \Delta t & a_2 \Delta t & a_2 \Delta t & -a_2 \Delta t & a_3 \Delta t \\
                \end{bmatrix}
            \end{aligned}
        \end{equation}


\ifx \allfiles \undefined    
\end{document}
\fi
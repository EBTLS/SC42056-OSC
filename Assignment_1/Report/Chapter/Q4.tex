\ifx \allfiles \undefined

\documentclass[titlepage,a4paper]{article}

\usepackage{amsmath}
\usepackage{float}
\usepackage{caption}
\usepackage{booktabs}
\usepackage{eurosym}
\usepackage[a4paper,text={170mm,257mm}]{geometry}



\begin{document}

\fi

    \setlength{\parindent}{0pt}
    \setlength{\parskip}{0.5em} 

    \subsection{Analysis}

    The model of the problem has been clearly stated in the problem document, so the key point is to transform the model into a matrix available in MATLAB. Define the variable vector as follows:
    $$
    x=
    \begin{bmatrix}
        \dot{q_{ac,1}} & \dot{q_{ac,2}} & \cdots &\dot{q_{ac,2160}} &T_{b,2} & T_{b,3} & \cdots & T_{b,2160}
    \end{bmatrix} 
    $$

    Considering the unit of $\Phi_k$ is \euro/kWh and the unit of $\dot{q_{ac,k}}$ is kW, so $\Delta T$ in $\sum_{k=1}^N \Phi_k \dot{q_{ac,k}}$ should be $\Delta t = 1 (h)$.

    There are two ways for finding the corresponding matrices $H$ and $C$ in the standard form of quadratic optimization problem.

    \begin{enumerate}
        \item Measure 1: from equation to matrices
        \begin{equation}
            \begin{aligned}
                \sum_{k=1}^N \Phi_k \dot{q_{ac,k}} \Delta t&+(0.1+E_2/10)(T_{B,k}-T_{ref})^2 = \sum_{k=1}^N \Phi_k \dot{q_{ac,k}}+\sum_{k=1}^N 1.4(T_k^2+T_{ref}^2-2T_r T_k) \\
                & =  \sum_{k=1}^N \Phi_k \dot{q_{ac,k}} -2.8 T_r T_k +\sum_{k=1}^N 1.4 T_{ref}^2 +\sum_{k=1}^N 1.4 T_k^2 \\
                & = \sum_{k=1}^N 1.4 T_{ref}^2+ C^T x + \frac{1}{2}x^T H x
            \end{aligned}
        \end{equation}
        
        so,

        \begin{equation}
            \begin{aligned}
                C=
                [\underbrace{\Phi_1 ~ \cdots ~ \cdots ~ \Phi_{2160}}_{N} ~ \underbrace{-2.8 T_{ref} ~ \cdots ~ \cdots ~  -2.8 T_{ref}}_{N-1} ]^T \\
                H=
                \begin{array}{c@{\hspace{-5pt}}c}
                    \overbrace{\hphantom{
                    \begin{array}{ccc} 
                        0 & \cdots & 0\\
                        \vdots & \ddots & \vdots \\
                        0 & \cdots & 0
                    \end{array}}}^{\displaystyle N}~~
                    \overbrace{\hphantom{
                    \begin{array}{ccc}
                        0 & \cdots & 0\\
                        \vdots & \ddots & \vdots \\
                        0 & \cdots & 0
                    \end{array}}}^{\displaystyle N-1}&\\
                    \left[
                        \begin{array}{c|c} 
                            \begin{array}{ccc} 
                                0 & \cdots & 0\\
                                \vdots & \ddots & \vdots \\
                                0 & \cdots & 0
                            \end{array} & 
                            \begin{array}{ccc}  
                                0 & \cdots & 0\\
                                \vdots & \ddots & \vdots \\
                                0 & \cdots & 0 
                            \end{array}\\
                            \hline
                            \begin{matrix}  
                                0 & \cdots & 0\\
                                \vdots & \ddots & \vdots \\
                                0 & \cdots & 0
                            \end{matrix} &
                            \begin{matrix}  
                                2.8 & \cdots & 0\\ 
                                \vdots & \ddots & \vdots \\
                                0 & \cdots & 2.8 
                            \end{matrix}
                        \end{array}
                    \right] &
                    \begin{array}{l}
                        \left.\vphantom{
                            \begin{array}{c} 
                                1 \\
                                \vdots \\
                                1 
                            \end{array}}
                            \right\}
                            N\\
                        \left.\vphantom{
                            \begin{array}{c} 
                                1 \\ 
                                \vdots \\
                                1 
                            \end{array}}
                            \right\}
                            N-1
                    \end{array}
                \end{array}
            \end{aligned}
        \end{equation}
        
        \item Measure 2: directly from matrices operation
        \begin{equation}
            \begin{aligned}
                \sum_{k=1}^N \Phi_k \dot{q_{ac,k}} \Delta t &+(0.1+E_2/10)(T_{B,k}-T_{ref})^2 = \sum_{k=1}^N \Phi_k \dot{q_{ac,k}}+\sum_{k=1}^N 1.4(T_k^2+T_{ref}^2-2T_r T_k) \\
                & =  (\sum_{k=1}^N \Phi_k \dot{q_{ac,k}}) -1.4(Ax-T_{mref})^T(Ax-T_{mref}) \\
                & = (\sum_{k=1}^N \Phi_k \dot{q_{ac,k}}) -1.4x^T A^T A x -1.4T_{mref}^T T_{mref} + 2.8 T_{mref}^T A x\\
                & = -2.8 T_{mref}^T T_{mref} + \Phi x  + 2.8 T_{mref}^T A x- 1.4 x A^T A x \\
            \end{aligned}
        \end{equation}

        where:

        \begin{equation}
            \begin{aligned}
                \Phi & = 
                [\underbrace{\Phi_1 ~ \cdots ~ \cdots ~ \Phi_{2160}}_{N} ~ \underbrace{0 ~ \cdots ~ \cdots ~  0}_{N-1} ] \\
                T_{mref} & =
                [\underbrace{0 ~ \cdots ~ \cdots ~ a0}_{N} ~ \underbrace{T_{ref} ~ \cdots ~ \cdots ~  T_{ref}}_{N-1} ] \\
                A & =
                \begin{array}{c@{\hspace{-5pt}}c}
                    \overbrace{\hphantom{
                    \begin{array}{ccc} 
                        0 & \cdots & 0\\
                        \vdots & \ddots & \vdots \\
                        0 & \cdots & 0
                    \end{array}}}^{\displaystyle N}~~
                    \overbrace{\hphantom{
                    \begin{array}{ccc}
                        0 & \cdots & 0\\
                        \vdots & \ddots & \vdots \\
                        0 & \cdots & 0
                    \end{array}}}^{\displaystyle N-1}&\\
                    \left[
                        \begin{array}{c|c} 
                            \begin{array}{ccc} 
                                0 & \cdots & 0\\
                                \vdots & \ddots & \vdots \\
                                0 & \cdots & 0
                            \end{array} & 
                            \begin{array}{ccc}  
                                0 & \cdots & 0\\
                                \vdots & \ddots & \vdots \\
                                0 & \cdots & 0 
                            \end{array}\\
                            \hline
                            \begin{matrix}  
                                0 & \cdots & 0\\
                                \vdots & \ddots & \vdots \\
                                0 & \cdots & 0
                            \end{matrix} &
                            \begin{matrix}  
                                1 & \cdots & 0\\ 
                                \vdots & \ddots & \vdots \\
                                0 & \cdots & 1 
                            \end{matrix}
                        \end{array}
                    \right] &
                    \begin{array}{l}
                        \left.\vphantom{
                            \begin{array}{c} 
                                1 \\
                                \vdots \\
                                1 
                            \end{array}}
                            \right\}
                            N\\
                        \left.\vphantom{
                            \begin{array}{c} 
                                1 \\ 
                                \vdots \\
                                1 
                            \end{array}}
                            \right\}
                            N-1
                    \end{array}
                \end{array}
            \end{aligned}
        \end{equation}

        so,

        \begin{equation}
            \begin{aligned}
                C= 
                \Phi+2T_{mref}^T 
                =[\underbrace{\Phi_1 ~ \cdots ~ \cdots ~ \Phi_{2160}}_{N} ~ \underbrace{-2.8 T_{ref} ~ \cdots ~ \cdots ~  -2.8 T_{ref}}_{N-1} ]^T \\
                H=
                2 A^T A
                = 
                \begin{array}{c@{\hspace{-5pt}}c}
                    \overbrace{\hphantom{
                    \begin{array}{ccc} 
                        0 & \cdots & 0\\
                        \vdots & \ddots & \vdots \\
                        0 & \cdots & 0
                    \end{array}}}^{\displaystyle N}~~
                    \overbrace{\hphantom{
                    \begin{array}{ccc}
                        0 & \cdots & 0\\
                        \vdots & \ddots & \vdots \\
                        0 & \cdots & 0
                    \end{array}}}^{\displaystyle N-1}&\\
                    \left[
                        \begin{array}{c|c} 
                            \begin{array}{ccc} 
                                0 & \cdots & 0\\
                                \vdots & \ddots & \vdots \\
                                0 & \cdots & 0
                            \end{array} & 
                            \begin{array}{ccc}  
                                0 & \cdots & 0\\
                                \vdots & \ddots & \vdots \\
                                0 & \cdots & 0 
                            \end{array}\\
                            \hline
                            \begin{matrix}  
                                0 & \cdots & 0\\
                                \vdots & \ddots & \vdots \\
                                0 & \cdots & 0
                            \end{matrix} &
                            \begin{matrix}  
                                2.8 & \cdots & 0\\ 
                                \vdots & \ddots & \vdots \\
                                0 & \cdots & 2.8 
                            \end{matrix}
                        \end{array}
                    \right] &
                    \begin{array}{l}
                        \left.\vphantom{
                            \begin{array}{c} 
                                1 \\
                                \vdots \\
                                1 
                            \end{array}}
                            \right\}
                            N\\
                        \left.\vphantom{
                            \begin{array}{c} 
                                1 \\ 
                                \vdots \\
                                1 
                            \end{array}}
                            \right\}
                            N-1
                    \end{array}
                \end{array}
            \end{aligned}
        \end{equation}
        
    \end{enumerate}

    For constraints $A_{eq} x=b_{eq}$, the matrices are shown 

    \begin{equation}
        \begin{aligned}
            A_{eq}=
            \begin{array}{c@{\hspace{10pt}}c@{\hspace{10pt}}c}
                \left[
                    \begin{array}{c|c|c} 
                        \begin{matrix}
                            a & 0 & \cdots & 0 \\
                            0 & a & \ddots & \vdots \\
                            \vdots & \ddots & \ddots & 0 \\
                            0 & \cdots & 0 & a \\
                        \end{matrix} & 
                        \begin{matrix}  
                            0 \\ 
                            \vdots \\
                            0 \\
                        \end{matrix} &
                        \begin{matrix}  
                            1 & 0 & \cdots&  & 0\\ 
                            -1 & \ddots&  \ddots & & \vdots \\
                            0 & \ddots &  \ddots & \ddots& \vdots \\
                            \vdots & \ddots & \ddots& \ddots& 0\\
                            0 & \cdots & 0 & -1 & 1 \\
                        \end{matrix}
                    \end{array}
                \right] \\
                \underbrace{\hphantom{
                    \begin{array}{cccc} 
                        a & 0 & \cdots & 0 \\
                        0 & a & \ddots & \vdots \\
                        \vdots & \ddots & \ddots & 0 \\
                        0 & \cdots & 0 & a \\
                    \end{array}}}_{\displaystyle N-1} 
                \underbrace{\hphantom{
                    \begin{array}{c}
                        0 \\ 
                        \vdots \\
                        0 \\
                    \end{array}}}_{\displaystyle 1}
                \underbrace{\hphantom{
                    \begin{array}{ccccc}
                        1 & 0 & \cdots&  & 0\\ 
                        -1 & \ddots&  \ddots & & \vdots \\
                        0 & \ddots &  \ddots & \ddots& \vdots \\
                        \vdots & \ddots & \ddots& \ddots& 0\\
                        0 & \cdots & 0 & -1 & 1 \\
                    \end{array}}}_{\displaystyle N-1} 
            \end{array} \\          
            b_{eq}=
            \begin{bmatrix}
                (a_1 \dot{q_{solar,1}}  + a_2 \dot{q_{occ,1}}  t -a_2 \dot{q_{vent,1}}+ a_3 T_{amb,1}) \Delta t +T_{b,1} \\ 
                \vdots \\ 
                (a_1 \dot{q_{solar,N}} + a_2 \dot{q_{occ,N}}  -a_2 \dot{q_{vent,N}} + a_3 T_{amb,N} )\Delta t                    
            \end{bmatrix} \\                
        \end{aligned}
    \end{equation}



    \subsection{Model}

    The model has been clearly explained in the problem document, and the model is explained by using matrices here. Because $1.4 T_{ref}^2$ is a constant in this problem, it was ignored in the standard form, it will be added after optimization with MATLAB.

    \subsection{Solution}

    Based on equations, the corresponding parameters in MATLAB function \verb quadprog ~are shown as follows:

    $$
    \begin{aligned}
        H=
        \begin{array}{c@{\hspace{-5pt}}c}
            \overbrace{\hphantom{
            \begin{array}{ccc} 
                0 & \cdots & 0\\
                \vdots & \ddots & \vdots \\
                0 & \cdots & 0
            \end{array}}}^{\displaystyle N}~~
            \overbrace{\hphantom{
            \begin{array}{ccc}
                0 & \cdots & 0\\
                \vdots & \ddots & \vdots \\
                0 & \cdots & 0
            \end{array}}}^{\displaystyle N-1}&\\
            \left[
                \begin{array}{c|c} 
                    \begin{array}{ccc} 
                        0 & \cdots & 0\\
                        \vdots & \ddots & \vdots \\
                        0 & \cdots & 0
                    \end{array} & 
                    \begin{array}{ccc}  
                        0 & \cdots & 0\\
                        \vdots & \ddots & \vdots \\
                        0 & \cdots & 0 
                    \end{array}\\
                    \hline
                    \begin{matrix}  
                        0 & \cdots & 0\\
                        \vdots & \ddots & \vdots \\
                        0 & \cdots & 0
                    \end{matrix} &
                    \begin{matrix}  
                        2.8 & \cdots & 0\\ 
                        \vdots & \ddots & \vdots \\
                        0 & \cdots & 2.8 
                    \end{matrix}
                \end{array}
            \right] &
            \begin{array}{l}
                \left.\vphantom{
                    \begin{array}{c} 
                        1 \\
                        \vdots \\
                        1 
                    \end{array}}
                    \right\}
                    N\\
                \left.\vphantom{
                    \begin{array}{c} 
                        1 \\ 
                        \vdots \\
                        1 
                    \end{array}}
                    \right\}
                    N-1
            \end{array}
        \end{array} \\
        f=
        [\underbrace{\Phi_1 ~ \cdots ~ \cdots ~ \Phi_{2160}}_{N} ~ \underbrace{-2.8 T_{ref} ~ \cdots ~ \cdots ~  -2.8 T_{ref}}_{N-1} ]^T  \\
        A_{eq}=
        \begin{array}{c@{\hspace{10pt}}c@{\hspace{10pt}}c}
            \left[
                \begin{array}{c|c|c} 
                    \begin{matrix}
                        a & 0 & \cdots & 0 \\
                        0 & a & \ddots & \vdots \\
                        \vdots & \ddots & \ddots & 0 \\
                        0 & \cdots & 0 & a \\
                    \end{matrix} & 
                    \begin{matrix}  
                        0 \\ 
                        \vdots \\
                        0 \\
                    \end{matrix} &
                    \begin{matrix}  
                        1 & 0 & \cdots&  & 0\\ 
                        -1 & \ddots&  \ddots & & \vdots \\
                        0 & \ddots &  \ddots & \ddots& \vdots \\
                        \vdots & \ddots & \ddots& \ddots& 0\\
                        0 & \cdots & 0 & -1 & 1 \\
                    \end{matrix}
                \end{array}
            \right] \\
            \underbrace{\hphantom{
                \begin{array}{cccc} 
                    a & 0 & \cdots & 0 \\
                    0 & a & \ddots & \vdots \\
                    \vdots & \ddots & \ddots & 0 \\
                    0 & \cdots & 0 & a \\
                \end{array}}}_{\displaystyle N-1} 
            \underbrace{\hphantom{
                \begin{array}{c}
                    0 \\ 
                    \vdots \\
                    0 \\
                \end{array}}}_{\displaystyle 1}
            \underbrace{\hphantom{
                \begin{array}{ccccc}
                    1 & 0 & \cdots&  & 0\\ 
                    -1 & \ddots&  \ddots & & \vdots \\
                    0 & \ddots &  \ddots & \ddots& \vdots \\
                    \vdots & \ddots & \ddots& \ddots& 0\\
                    0 & \cdots & 0 & -1 & 1 \\
                \end{array}}}_{\displaystyle N-1} 
        \end{array} \\ 
        b_{eq}=
        \begin{bmatrix}
            (a_1 \dot{q_{solar,1}}  + a_2 \dot{q_{occ,1}}  t -a_2 \dot{q_{vent,1}}+ a_3 T_{amb,1}) \Delta t +T_{b,1}\\ 
            \vdots \\ 
                (a_1 \dot{q_{solar,N}} + a_2 \dot{q_{occ,N}}  -a_2 \dot{q_{vent,N}} + a_3 T_{amb,N} )\Delta t                    
        \end{bmatrix} \\
        lb(i)=
        \begin{cases}
            0  & \text{i=1,...,N} \\
            15 & \text{i > 2160 and $ \dot{q_{occ,(i-2160)}} >0 $} \\
            -\inf & \text (others)
        \end{cases} \\
        ub(i)=
        \begin{cases}
            \dot{q_{ac,max}}  & \text{i=1,...,N} \\
            28 & \text{i > 2160 and $ \dot{q_{occ,(i-2160)}} >0 $} \\
            \inf & \text (others) \\
        \end{cases} 
    \end{aligned}
    $$


    Finally, the quadprog got the optimal answer, which obtained the minimal cost of \euro 20533.

    
    \subsection{Answer}

    1. The optimal cost for air-conditioning along the horizon of N(2160) steps is \euro 20533.



\ifx \allfiles \undefined    
\end{document}
\fi
\ifx \allfiles \undefined

\documentclass[titlepage,a4paper]{article}

\usepackage{amsmath}
\usepackage{float}
\usepackage{caption}
\usepackage{booktabs}
\usepackage{eurosym}
\usepackage[a4paper]{geometry}



\begin{document}

\fi

    \setlength{\parindent}{0pt}
    \setlength{\parskip}{0.5em}

    \subsection{Question 4}
    

        \subsubsection{Assumption}





        \subsubsection{Analysis}

        The model of this problem is clearly clarified in the problem document, so the main point is to transform this model to available matrix for MATLAB. Define the variable vector as follows:
        $$
        \vec{x_4}=
        \begin{bmatrix}
            \dot{q_{ac,1}} & \dot{q_{ac,2}} & \cdots &\dot{q_{ac,2160}} &T_{b,2} & T_{b,3} & \cdots & T_{b,2160}
        \end{bmatrix} 
        $$

        Considering the unit of $\Phi_k$ is \euro/kWh and the unit of $\dot{q_{ac,k}}$ is kW, so $\Delta t$ in $\sum_{k=1}^N \Phi_k \dot{q_{ac,k}}$ should be $\Delta t = 1 (h)$.

        There are two ways for finding the corresponding matrices $H$ and $C$ in the standard form of quadratic optimization problem.

        \begin{enumerate}
            \item Measure 1: from equation to matrices
            \begin{equation}
                \begin{aligned}
                    \sum_{k=1}^N \Phi_k \dot{q_{ac,k}} \Delta t+(0.1+E_2/10)(T_{B,k}-T_{ref})^2 &= \sum_{k=1}^N \Phi_k \dot{q_{ac,k}}+\sum_{k=1}^N 1.4(T_k^2+T_{ref}^2-2T_r T_k) \\
                    & =  \sum_{k=1}^N \Phi_k \dot{q_{ac,k}} -2.8 T_r T_k +\sum_{k=1}^N 1.4 T_{ref}^2 +\sum_{k=1}^N 1.4 T_k^2 \\
                    & = \sum_{k=1}^N 1.4 T_{ref}^2+ C^T \vec{x_4} + \frac{1}{2}\vec{x_4}^T H \vec{x_4}
                \end{aligned}
            \end{equation}
            
            so,

            \begin{equation}
                \begin{aligned}
                    C=
                    [\underbrace{\Phi_1 ~ \cdots ~ \cdots ~ \Phi_{2160}}_{N} ~ \underbrace{-2.8 T_{ref} ~ \cdots ~ \cdots ~  -2.8 T_{ref}}_{N-1} ]^T \\
                    H=
                    \begin{array}{c@{\hspace{-5pt}}c}
                        \overbrace{\hphantom{
                        \begin{array}{ccc} 
                            0 & \cdots & 0\\
                            \vdots & \ddots & \vdots \\
                            0 & \cdots & 0
                        \end{array}}}^{\displaystyle N}~~
                        \overbrace{\hphantom{
                        \begin{array}{ccc}
                            0 & \cdots & 0\\
                            \vdots & \ddots & \vdots \\
                            0 & \cdots & 0
                        \end{array}}}^{\displaystyle N-1}&\\
                        \left[
                            \begin{array}{c|c} 
                                \begin{array}{ccc} 
                                    0 & \cdots & 0\\
                                    \vdots & \ddots & \vdots \\
                                    0 & \cdots & 0
                                \end{array} & 
                                \begin{array}{ccc}  
                                    0 & \cdots & 0\\
                                    \vdots & \ddots & \vdots \\
                                    0 & \cdots & 0 
                                \end{array}\\
                                \hline
                                \begin{matrix}  
                                    0 & \cdots & 0\\
                                    \vdots & \ddots & \vdots \\
                                    0 & \cdots & 0
                                \end{matrix} &
                                \begin{matrix}  
                                    2.8 & \cdots & 0\\ 
                                    \vdots & \ddots & \vdots \\
                                    0 & \cdots & 2.8 
                                \end{matrix}
                            \end{array}
                        \right] &
                        \begin{array}{l}
                            \left.\vphantom{
                                \begin{array}{c} 
                                    1 \\
                                    \vdots \\
                                    1 
                                \end{array}}
                                \right\}
                                N\\
                            \left.\vphantom{
                                \begin{array}{c} 
                                    1 \\ 
                                    \vdots \\
                                    1 
                                \end{array}}
                                \right\}
                                N-1
                        \end{array}
                    \end{array}
                \end{aligned}
            \end{equation}
            
            \item Measure 2: directly from matrices operation
            \begin{equation}
                \begin{aligned}
                    \sum_{k=1}^N \Phi_k \dot{q_{ac,k}} \Delta t+(0.1+E_2/10)(T_{B,K}-T_{ref})^2 &= \sum_{k=1}^N \Phi_k \dot{q_{ac,k}}+\sum_{k=1}^N 1.4(T_k^2+T_{ref}^2-2T_r T_k) \\
                    & =  (\sum_{k=1}^N \Phi_k \dot{q_{ac,k}}) -1.4(Ax-T_{mref})^T(Ax-T_{mref}) \\
                    & = (\sum_{k=1}^N \Phi_k \dot{q_{ac,k}}) -1.4x^T A^T A X -1.4T_{mref}^T T_{mref} + 2.8 T_{mref}^T A x\\
                    & = -2.8 T_{mref}^T T_{mref} + \Phi x  + 2.8 T_{mref}^T A x- 1.4^T A^T A X \\
                \end{aligned}
            \end{equation}

            where:

            \begin{equation}
                \begin{aligned}
                    \Phi & = 
                    [\underbrace{\Phi_1 ~ \cdots ~ \cdots ~ \Phi_{2160}}_{N} ~ \underbrace{0 ~ \cdots ~ \cdots ~  0}_{N-1} ] \\
                    T_{mref} & =
                    [\underbrace{0 ~ \cdots ~ \cdots ~ a0}_{N} ~ \underbrace{T_{ref} ~ \cdots ~ \cdots ~  T_{ref}}_{N-1} ] \\
                    A & =
                    \begin{array}{c@{\hspace{-5pt}}c}
                        \overbrace{\hphantom{
                        \begin{array}{ccc} 
                            0 & \cdots & 0\\
                            \vdots & \ddots & \vdots \\
                            0 & \cdots & 0
                        \end{array}}}^{\displaystyle N}~~
                        \overbrace{\hphantom{
                        \begin{array}{ccc}
                            0 & \cdots & 0\\
                            \vdots & \ddots & \vdots \\
                            0 & \cdots & 0
                        \end{array}}}^{\displaystyle N-1}&\\
                        \left[
                            \begin{array}{c|c} 
                                \begin{array}{ccc} 
                                    0 & \cdots & 0\\
                                    \vdots & \ddots & \vdots \\
                                    0 & \cdots & 0
                                \end{array} & 
                                \begin{array}{ccc}  
                                    0 & \cdots & 0\\
                                    \vdots & \ddots & \vdots \\
                                    0 & \cdots & 0 
                                \end{array}\\
                                \hline
                                \begin{matrix}  
                                    0 & \cdots & 0\\
                                    \vdots & \ddots & \vdots \\
                                    0 & \cdots & 0
                                \end{matrix} &
                                \begin{matrix}  
                                    1 & \cdots & 0\\ 
                                    \vdots & \ddots & \vdots \\
                                    0 & \cdots & 1 
                                \end{matrix}
                            \end{array}
                        \right] &
                        \begin{array}{l}
                            \left.\vphantom{
                                \begin{array}{c} 
                                    1 \\
                                    \vdots \\
                                    1 
                                \end{array}}
                                \right\}
                                N\\
                            \left.\vphantom{
                                \begin{array}{c} 
                                    1 \\ 
                                    \vdots \\
                                    1 
                                \end{array}}
                                \right\}
                                N-1
                        \end{array}
                    \end{array}
                \end{aligned}
            \end{equation}

            so,

            \begin{equation}
                \begin{aligned}
                    C= 
                    \Phi+2T_{mref}^T 
                    =[\underbrace{\Phi_1 ~ \cdots ~ \cdots ~ \Phi_{2160}}_{N} ~ \underbrace{-2.8 T_{ref} ~ \cdots ~ \cdots ~  -2.8 T_{ref}}_{N-1} ]^T \\
                    H=
                    2 A^T A
                    = 
                    \begin{array}{c@{\hspace{-5pt}}c}
                        \overbrace{\hphantom{
                        \begin{array}{ccc} 
                            0 & \cdots & 0\\
                            \vdots & \ddots & \vdots \\
                            0 & \cdots & 0
                        \end{array}}}^{\displaystyle N}~~
                        \overbrace{\hphantom{
                        \begin{array}{ccc}
                            0 & \cdots & 0\\
                            \vdots & \ddots & \vdots \\
                            0 & \cdots & 0
                        \end{array}}}^{\displaystyle N-1}&\\
                        \left[
                            \begin{array}{c|c} 
                                \begin{array}{ccc} 
                                    0 & \cdots & 0\\
                                    \vdots & \ddots & \vdots \\
                                    0 & \cdots & 0
                                \end{array} & 
                                \begin{array}{ccc}  
                                    0 & \cdots & 0\\
                                    \vdots & \ddots & \vdots \\
                                    0 & \cdots & 0 
                                \end{array}\\
                                \hline
                                \begin{matrix}  
                                    0 & \cdots & 0\\
                                    \vdots & \ddots & \vdots \\
                                    0 & \cdots & 0
                                \end{matrix} &
                                \begin{matrix}  
                                    2.8 & \cdots & 0\\ 
                                    \vdots & \ddots & \vdots \\
                                    0 & \cdots & 2.8 
                                \end{matrix}
                            \end{array}
                        \right] &
                        \begin{array}{l}
                            \left.\vphantom{
                                \begin{array}{c} 
                                    1 \\
                                    \vdots \\
                                    1 
                                \end{array}}
                                \right\}
                                N\\
                            \left.\vphantom{
                                \begin{array}{c} 
                                    1 \\ 
                                    \vdots \\
                                    1 
                                \end{array}}
                                \right\}
                                N-1
                        \end{array}
                    \end{array}
                \end{aligned}
            \end{equation}
            
        \end{enumerate}

        For constraints $A_{eq} x=b_{eq}$, the matrices are shown 

        % \begin{equation}
        %     \begin{aligned}
        %         A_{eq}=
        %         [
        %         \underbrace{
        %             \begin{matrix}
        %                 -a_2 \Delta t & 0 & \codts & 0 \\
        %                 0 & -a_2 \Delta t & \cdots & 0 \\
        %                 \vdots & \vdots & \ddots & \vdots \\
        %                 0 & \cdots & -a_2 \Delta t & 0 \\
        %                 0 & \cdots & 0 & -a_2 \Delta t \\
        %             \end{matrix}}_{N} ~ 
        %         \underbrace{
        %             \begin{matrix}
        %                 0 \\
        %                 0 \\
        %                 \vdots \\
        %                 0 \\
        %                 0 \\
        %             \end{matrix}
        %             }_{1}
        %         \underbrace{
        %             \begin{matrix}
        %                 1 & 0 & \cdots & 0 \\
        %                 -(1-a_3 \Delta t) & 1 & \ddots & \vdots \\
        %                 0 & \ddots & \ddots & \\
        %                 \vdots & \ddots &  & \\
        %                 0 & 0 & -(1-a_3 \Delta t)
        %             \end{matrix}}_{N-1} ] \\
        %     \end{aligned}
        % \end{equation}

        \subsubsection{Model}

        The model has been clearly clarified in the problem document, here the model is explained using matrices. Because $1.4 T_{ref}^2$ is a constant in this problem,so it was ignored in the standard form, and will finally be added on after calculating with MATLAB.

        \subsubsection{Solution}

        Based on equations, the corresponding parameters in MATLAB function \verb quadprog ~are shown as follows:

        \begin{equation}
            \begin{aligned}
                H=
                \begin{array}{c@{\hspace{-5pt}}c}
                    \overbrace{\hphantom{
                    \begin{array}{ccc} 
                        0 & \cdots & 0\\
                        \vdots & \ddots & \vdots \\
                        0 & \cdots & 0
                    \end{array}}}^{\displaystyle N}~~
                    \overbrace{\hphantom{
                    \begin{array}{ccc}
                        0 & \cdots & 0\\
                        \vdots & \ddots & \vdots \\
                        0 & \cdots & 0
                    \end{array}}}^{\displaystyle N-1}&\\
                    \left[
                        \begin{array}{c|c} 
                            \begin{array}{ccc} 
                                0 & \cdots & 0\\
                                \vdots & \ddots & \vdots \\
                                0 & \cdots & 0
                            \end{array} & 
                            \begin{array}{ccc}  
                                0 & \cdots & 0\\
                                \vdots & \ddots & \vdots \\
                                0 & \cdots & 0 
                            \end{array}\\
                            \hline
                            \begin{matrix}  
                                0 & \cdots & 0\\
                                \vdots & \ddots & \vdots \\
                                0 & \cdots & 0
                            \end{matrix} &
                            \begin{matrix}  
                                2.8 & \cdots & 0\\ 
                                \vdots & \ddots & \vdots \\
                                0 & \cdots & 2.8 
                            \end{matrix}
                        \end{array}
                    \right] &
                    \begin{array}{l}
                        \left.\vphantom{
                            \begin{array}{c} 
                                1 \\
                                \vdots \\
                                1 
                            \end{array}}
                            \right\}
                            N\\
                        \left.\vphantom{
                            \begin{array}{c} 
                                1 \\ 
                                \vdots \\
                                1 
                            \end{array}}
                            \right\}
                            N-1
                    \end{array}
                \end{array} \\
                f=
                =[\underbrace{\Phi_1 ~ \cdots ~ \cdots ~ \Phi_{2160}}_{N} ~ \underbrace{-2.8 T_{ref} ~ \cdots ~ \cdots ~  -2.8 T_{ref}}_{N-1} ]^T  \\
                % A_{eq}=
                % b_{eq}=
            \end{aligned}
        \end{equation}
        \begin{equation}
            \begin{aligned}
                lb(i)=
                \begin{cases}
                   0  & \text{i=1,...,N} \\
                   15 & \text{i > 2160 and $ \dot{q_{occ,(i-2160)}} >0 $} \\
                   -\inf & \text (others)
                \end{cases} \\
                ub(i)=
                \begin{cases}
                   \dot{q_{ac,max}}  & \text{i=1,...,N} \\
                   28 & \text{i > 2160 and $ \dot{q_{occ,(i-2160)}} >0 $} \\
                   \inf & \text (others)
                \end{cases}
            \end{aligned}
        \end{equation}

        Finally, the quadprog got the optimal answer, which obtain the minimal cost of \euro 20533.




        \subsubsection{Answer}
        1. The optimal cost for air-conditioning along the horizon of N(2160) steps is \euro 20533.



\ifx \allfiles \undefined    
\end{document}
\fi